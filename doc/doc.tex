\documentclass[12pt, a5paper]{article}

\usepackage[utf8]{inputenc}
\usepackage[T1]{fontenc}
\usepackage{graphicx}
\usepackage{longtable}
\usepackage{float}
\usepackage{wrapfig}
\usepackage{soul}
\usepackage{amssymb}
\usepackage{hyperref}
\parindent 0in
\usepackage[spanish]{babel}
\setlength{\parskip}{0.5\baselineskip}
\usepackage{multirow}
\usepackage{amsthm}
\usepackage{amsmath}
\usepackage{fancyhdr}
\usepackage[twoside, margin=0.3in, bottom=0in, inner=0.5in, includefoot,heightrounded]{geometry}
\usepackage{multicol}

\fancyhf{} % clear all header and footers
\renewcommand{\headrulewidth}{0pt} % remove the header rule
\renewcommand{\footrulewidth}{0pt}
\setlength\footskip{10pt}
\fancyfoot[LO, RE]{\thepage}

\setlength{\columnsep}{0.3in}

\newtheorem*{teo}{\emph{$T^a$}}
\newtheorem*{prop}{Prop}

\theoremstyle{definition}
\newtheorem*{ejem}{Ejemplo}
\newtheorem*{defi}{Def}

\usepackage{listings}
\usepackage{color}

\definecolor{mygreen}{rgb}{0,0.6,0}
\definecolor{mygray}{rgb}{0.5,0.5,0.5}
\definecolor{mymauve}{rgb}{0.58,0,0.82}
\lstset{
  columns=fullflexible,
  backgroundcolor=\color{white},   % choose the background color; you must add \usepackage{color} or \usepackage{xcolor}
  basicstyle=\ttfamily\footnotesize,        % the size of the fonts that are used for the code
  breakatwhitespace=false,         % sets if automatic breaks should only happen at whitespace
  breaklines=true,                 % sets automatic line breaking
  captionpos=b,                    % sets the caption-position to bottom
  commentstyle=\color{mygreen},    % comment style
  %deletekeywords={...},            % if you want to delete keywords from the given language
  inputencoding=utf8,
  %escapeinside={\%*}{*)},          % if you want to add LaTeX within your code
  extendedchars=true,              % lets you use non-ASCII characters; for 8-bits encodings only, does not work with UTF-8
  literate= {á}{{\'a}}1 {α}{{$\alpha$}}1 {β}{{$\beta$}}1 {é}{{\'e}}1 
      {í}{{\'i}}1 {ó}{{\'o}}1 {ú}{{\'u}}1 {ñ}{{\~n}}1
			{Á}{{\'A}}1 {É}{{\'E}}1 {Í}{{\'I}}1 {Ó}{{\'O}}1 {Ú}{{\'U}}1
      {Ñ}{{\~N}}1 {δ}{{$\delta$}}1 {ε}{{$\epsilon$}}1
			{_}{{\_}}1 {^}{{\textasciicircum}}1,
  frame=single,                    % adds a frame around the code
  keepspaces=true,                 % keeps spaces in text, useful for keeping indentation of code (possibly needs columns=flexible)
  keywordstyle=\color{blue},       % keyword style
  language=haskell,                 % the language of the code
  morekeywords={ll,ii,vi,vii,vvi,vll,mii,ld,point,vect,line,circle,polygon},
            % if you want to add more keywords to the set
  numbers=left,                    % where to put the line-numbers; possible values are (none, left, right)
  numbersep=5pt,                   % how far the line-numbers are from the code
  numberstyle=\tiny\color{mygray}, % the style that is used for the line-numbers
  rulecolor=\color{black},         % if not set, the frame-color may be changed on line-breaks within not-black text (e.g. comments (green here))
  showspaces=false,                % show spaces everywhere adding particular underscores; it overrides 'showstringspaces'
  showstringspaces=false,          % underline spaces within strings only
  showtabs=false,                  % show tabs within strings adding particular underscores
  stepnumber=1,                    % the step between two line-numbers. If it's 1, each line will be numbered
  stringstyle=\color{mymauve},     % string literal style
  tabsize=4,                       % sets default tabsize to 2 spaces
  %title=\lstname,                   % show the filename of files included with \lstinputlisting; also try caption instead of title
  texcl=true
}

\pagestyle{fancy}

\begin{document}
\input{portada.tex}
\pagenumbering{gobble}
\pagenumbering{arabic}
\tableofcontents
\newpage
%\section{DPletario}
%\lstinputlisting[firstline=10]{dpletario.cpp}
%\newpage
\section{Introducción}
El objetivo de este documento es complementar mi práctica de ALCP.
Esto es así porque no me gusta poner comentarios excesivos en mi
código, así que prefiero dejar las explicaciones a parte.

He hecho la práctica en Haskell, porque es un paradigma de
programación mucho más cercano a los algoritmos que debemos
implementar. Esto permite en algunos puntos un código muy limpio.

A cambio, ha sido un poco engorroso tener que pasar
explícitamente las estructuras algebraicas sobre las que se trabaja
cada vez.

\newpage

\section{Definitions}
El primer módulo relevante es Definitions. Aquí se definen las
estructuras y algunos algoritmos básicos que se usarán en toda la
práctica.

En primer lugar, defino una estructura Dictionary, que contiene las
operaciones de cada estructura algebraica. He definido cuerpos,
dominios Euclídeos y dominios de factorización única. Para los
dominios euclídeos está definida una división que sólo se garantiza
que funciona al dividir por una unidad (O sea, cuando el grado es 1).

En este módulo también está el primer algoritmo importante: el
algoritmo de Euclides extendido. Es importante que está definido para
cualquier dominio Euclídeo. También hay algunos algoritmos similares
derivados.

También en Definitions está la exponenciación rápida en cualquier
anillo, y una pequeña función para calcular el orden de un elemento de
cualquier anillo.
\newpage\lstinputlisting[firstline=7]{../Definitions.hs}\newpage

\section{Utilities}
Utilities tiene funciones de ``fontanería'' de Haskell que he
preferido redefinir.

También tiene una criba de Eratóstenes muy refinada, (Utilizando
rotaciones y filtros sucesivos, junto con la evaluación perezosa).
Usando esta criba, he implementado un pequeño algoritmo de
factorización de enteros.
\newpage\lstinputlisting[firstline=1]{../Utilities.hs}\newpage

\section{Numbers y Fract}
Los dos primeros son  módulos sencillos que no requieren mucha explicación.
Uno representa a los enteros (de precisión arbitraria) como dominio
euclídeo. Otro representa el cuerpo de fracciones asociado a un
dominio euclídeo.

En cuanto a Gauss, representa a los enteros de Gauss como dominio
euclídeo, usando como grado la norma euclídea. Para la división, hago
la división de los complejos usual y busco de las cuatro opciones
resultantes de sumar o no una unidad a cada coordenada, la que está
mas cerca.

Esto automáticamente resuelve el EEA para los enteros de Gauss. Es la
ventaja de la modularidad de mi práctica.

\newpage\lstinputlisting[firstline=5]{../Numbers.hs}
\lstinputlisting[firstline=6]{../Fract.hs}
\lstinputlisting[firstline=7]{../Gauss.hs}

\newpage

\section{Quotient}
Quotient define la operación de tomar el módulo de un dominio euclídeo
por un ideal generado por un elemento. Las operaciones son simplemente
tomar restos tras cualquier operación. Para la inversa, se aplica el
EEA para calcular el t válido. Es muy importante que esta versión
puede calcular divisiones cuando estas son posibles. Para ello,
manipulo un poco los resultados que da el EEA para que se resuelva la
ecuación en general.

Este módulo en particular es lo que hará falta más adelante para definir los
cuerpos finitos. En este caso, se toma $\mathbb{F}_p[x]/<f>$ donde f es un polinomio
irreducible. 

También se define un resoluto del Teorema del resto chino. Éste
funciona en cualquier dominio euclídeo cuando los módulos son primos
entre sí.

He implementado, para el caso de los enteros, un resolutor que permite
sistemas en que los módulos no son primos entre sí. Para esto, 
chineseSplit factoriza los módulos, descompone cada ecuación en un
conjunto de ecuaciones de módulo potencia de un primo. Luego,
chineseFilter elimina las ecuaciones redundantes comprobando que son
coherentes. Por último, se resuelve el sistema con el resolutor
general.

El motivo por el que implemento este resolutor es para resolver más
adelante el logaritmo discreto por Pollard-rho.

\newpage\lstinputlisting[firstline=6]{../Quotient.hs}\newpage

\section{Polynomials}
Polynomials es, obviamente, uno de los módulos más básicos de la
práctica. Genera el domino euclídeo de polinomios asociado a un
cuerpo. También está definido para dominios euclídeos, para poder más
adelante implementar la factorización en $\mathbb{Z}$.

Básicamente se implementan las operaciones con los algoritmos
elementales. Los polinomios se modelizan como listas de coeficientes,
de mayor a menor grado. 

Un detalle relevante es que para los polinomios, la función deg
devuelve una unidad menos que el grado del polinomio. Esto es porque
el grado como dominio euclídeo es diferente al grado del polinomio.
Para mantener la práctica coherente con los algoritmos, aquí el grado
es el grado euclídeo.

También se implementan algunas funciones que serán útiles más
adelante, como puede ser la reducción de un polinomio a su equivalente
mónico o la derivada.

\newpage\lstinputlisting{../Polynomials.hs}\newpage


\clearpage
\null
\pagenumbering{gobble}
\clearpage
\begin{table}
\centering
\begin{tabular}{c}
\emph{Alejandro Aguirre Galindo} \\
\emph{Luis María Costero Valero} \\
\emph{Jesús Doménech Arellano} \\
\end{tabular}
\end{table}
\end{document}
